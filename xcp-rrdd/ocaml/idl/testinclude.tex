\newpage\section{Session Management}

\subsection{Session.Login}

{\bf Overview:} 
Attempt to authenticate the user, returning a session\_id if successful

 \noindent {\bf Signature:} 
\begin{verbatim} session_id Session.Login (string uname, string pwd)\end{verbatim}


\noindent{\bf Arguments:}

 
\vspace{0.3cm}
\begin{tabular}{|c|c|p{7cm}|}
 \hline
{\bf type} & {\bf name} & {\bf description} \\ \hline
{\tt string } & uname & Username for login. \\ \hline 

{\tt string } & pwd & Password for login. \\ \hline 

\end{tabular}

\vspace{0.3cm}

 \noindent {\bf Return Type:} 
{\tt 
session\_id
}


ID of newly created session
\vspace{0.3cm}
\vspace{0.3cm}
\vspace{0.3cm}

\subsection{Session.Logout}

{\bf Overview:} 
Log out of a session

 \noindent {\bf Signature:} 
\begin{verbatim} void Session.Logout (session_id s)\end{verbatim}


\vspace{0.3cm}

 \noindent {\bf Return Type:} 
{\tt 
void
}



\vspace{0.3cm}
\vspace{0.3cm}
\vspace{0.3cm}



\newpage\section{VM Management}

\subsection{VM.Install}

{\bf Overview:} 
Install a new VM

 \noindent {\bf Signature:} 
\begin{verbatim} vm_id VM.Install (session_id s, vm_config config)\end{verbatim}


\noindent{\bf Arguments:}

 
\vspace{0.3cm}
\begin{tabular}{|c|c|p{7cm}|}
 \hline
{\bf type} & {\bf name} & {\bf description} \\ \hline
{\tt vm\_config } & config & An XML String specifying initial configuration for the VM. \\ \hline 

\end{tabular}

\vspace{0.3cm}

 \noindent {\bf Return Type:} 
{\tt 
vm\_id
}


The ID of the installed VM.
\vspace{0.3cm}
\vspace{0.3cm}
\vspace{0.3cm}

\subsection{VM.GetConfig}

{\bf Overview:} 
Return the XML Config file specifying the current state of a VM.

 \noindent {\bf Signature:} 
\begin{verbatim} vm_config VM.GetConfig (session_id s, vm_id vm)\end{verbatim}


\noindent{\bf Arguments:}

 
\vspace{0.3cm}
\begin{tabular}{|c|c|p{7cm}|}
 \hline
{\bf type} & {\bf name} & {\bf description} \\ \hline
{\tt vm\_id } & vm & The VM whose config is being requested. \\ \hline 

\end{tabular}

\vspace{0.3cm}

 \noindent {\bf Return Type:} 
{\tt 
vm\_config
}


The XML Config file for the specified VM.
\vspace{0.3cm}
\vspace{0.3cm}
\vspace{0.3cm}

\subsection{VM.Delete}

{\bf Overview:} 
Delete the VM and its associated configuration and disk images from storage

 \noindent {\bf Signature:} 
\begin{verbatim} void VM.Delete (session_id s, vm_id vm)\end{verbatim}


\noindent{\bf Arguments:}

 
\vspace{0.3cm}
\begin{tabular}{|c|c|p{7cm}|}
 \hline
{\bf type} & {\bf name} & {\bf description} \\ \hline
{\tt vm\_id } & vm & The VM to delete \\ \hline 

\end{tabular}

\vspace{0.3cm}

 \noindent {\bf Return Type:} 
{\tt 
void
}



\vspace{0.3cm}
\vspace{0.3cm}
\vspace{0.3cm}

\subsection{VM.SetUnstartable}

{\bf Overview:} 
Prevents the VM from being started.
        This is useful for making template-VMs that are only used for cloning.
        (This function can only be called when the specified VM is in the Halted State).

 \noindent {\bf Signature:} 
\begin{verbatim} void VM.SetUnstartable (session_id s, vm_id vm)\end{verbatim}


\noindent{\bf Arguments:}

 
\vspace{0.3cm}
\begin{tabular}{|c|c|p{7cm}|}
 \hline
{\bf type} & {\bf name} & {\bf description} \\ \hline
{\tt vm\_id } & vm & The VM to set as unstartable \\ \hline 

\end{tabular}

\vspace{0.3cm}

 \noindent {\bf Return Type:} 
{\tt 
void
}



\vspace{0.3cm}
\vspace{0.3cm}
\vspace{0.3cm}

\subsection{VM.ClearUnstartable}

{\bf Overview:} 
Undoes the effect of VM.SetUnstartable, allowing a VM to be started subsequently.
        (This function can only be called when the VM is in the Halted State).

 \noindent {\bf Signature:} 
\begin{verbatim} void VM.ClearUnstartable (session_id s, vm_id vm)\end{verbatim}


\noindent{\bf Arguments:}

 
\vspace{0.3cm}
\begin{tabular}{|c|c|p{7cm}|}
 \hline
{\bf type} & {\bf name} & {\bf description} \\ \hline
{\tt vm\_id } & vm & The VM to set as unstartable \\ \hline 

\end{tabular}

\vspace{0.3cm}

 \noindent {\bf Return Type:} 
{\tt 
void
}



\vspace{0.3cm}
\vspace{0.3cm}
\vspace{0.3cm}

\subsection{VM.Clone}

{\bf Overview:} 
Clones the specified VM, making a new VM.
        (This function can only be called when the VM is in the Halted State).

 \noindent {\bf Signature:} 
\begin{verbatim} vm_id VM.Clone (session_id s, vm_id vm, string new_name)\end{verbatim}


\noindent{\bf Arguments:}

 
\vspace{0.3cm}
\begin{tabular}{|c|c|p{7cm}|}
 \hline
{\bf type} & {\bf name} & {\bf description} \\ \hline
{\tt vm\_id } & vm & The VM to be cloned \\ \hline 

{\tt string } & new\_name & The name of the cloned VM \\ \hline 

\end{tabular}

\vspace{0.3cm}

 \noindent {\bf Return Type:} 
{\tt 
vm\_id
}


The ID of the newly created VM.
\vspace{0.3cm}
\vspace{0.3cm}
\vspace{0.3cm}

\subsection{VM.Start}

{\bf Overview:} 
Start the specified VM.
        (This function can only be called with the VM is in the Halted State).

 \noindent {\bf Signature:} 
\begin{verbatim} void VM.Start (session_id s, vm_id vm)\end{verbatim}


\noindent{\bf Arguments:}

 
\vspace{0.3cm}
\begin{tabular}{|c|c|p{7cm}|}
 \hline
{\bf type} & {\bf name} & {\bf description} \\ \hline
{\tt vm\_id } & vm & The VM to start \\ \hline 

\end{tabular}

\vspace{0.3cm}

 \noindent {\bf Return Type:} 
{\tt 
void
}



\vspace{0.3cm}
\vspace{0.3cm}
\vspace{0.3cm}

\subsection{VM.Shutdown}

{\bf Overview:} 
Cleanly shutdown the specified VM.

 \noindent {\bf Signature:} 
\begin{verbatim} void VM.Shutdown (session_id s, vm_id vm)\end{verbatim}


\noindent{\bf Arguments:}

 
\vspace{0.3cm}
\begin{tabular}{|c|c|p{7cm}|}
 \hline
{\bf type} & {\bf name} & {\bf description} \\ \hline
{\tt vm\_id } & vm & The VM to shutdown \\ \hline 

\end{tabular}

\vspace{0.3cm}

 \noindent {\bf Return Type:} 
{\tt 
void
}



\vspace{0.3cm}
\vspace{0.3cm}
\vspace{0.3cm}

\subsection{VM.Destroy}

{\bf Overview:} 
Destroy the specified VM without attempting a clean shutdown

 \noindent {\bf Signature:} 
\begin{verbatim} void VM.Destroy (session_id s, vm_id vm)\end{verbatim}


\noindent{\bf Arguments:}

 
\vspace{0.3cm}
\begin{tabular}{|c|c|p{7cm}|}
 \hline
{\bf type} & {\bf name} & {\bf description} \\ \hline
{\tt vm\_id } & vm & The VM to destroy \\ \hline 

\end{tabular}

\vspace{0.3cm}

 \noindent {\bf Return Type:} 
{\tt 
void
}



\vspace{0.3cm}
\vspace{0.3cm}
\vspace{0.3cm}

\subsection{VM.Reboot}

{\bf Overview:} 
Reboot the specified VM

 \noindent {\bf Signature:} 
\begin{verbatim} void VM.Reboot (session_id s, vm_id vm)\end{verbatim}


\noindent{\bf Arguments:}

 
\vspace{0.3cm}
\begin{tabular}{|c|c|p{7cm}|}
 \hline
{\bf type} & {\bf name} & {\bf description} \\ \hline
{\tt vm\_id } & vm & The VM to reboot \\ \hline 

\end{tabular}

\vspace{0.3cm}

 \noindent {\bf Return Type:} 
{\tt 
void
}



\vspace{0.3cm}
\vspace{0.3cm}
\vspace{0.3cm}

\subsection{VM.Hibernate}

{\bf Overview:} 
Hibernate the specified VM

 \noindent {\bf Signature:} 
\begin{verbatim} void VM.Hibernate (session_id s, vm_id vm, bool live)\end{verbatim}


\noindent{\bf Arguments:}

 
\vspace{0.3cm}
\begin{tabular}{|c|c|p{7cm}|}
 \hline
{\bf type} & {\bf name} & {\bf description} \\ \hline
{\tt vm\_id } & vm & The VM to hibernate \\ \hline 

{\tt bool } & live & If set to true, perform a live hibernate; otherwise pause the VM before commencing hibernate \\ \hline 

\end{tabular}

\vspace{0.3cm}

 \noindent {\bf Return Type:} 
{\tt 
void
}



\vspace{0.3cm}
\vspace{0.3cm}
\vspace{0.3cm}

\subsection{VM.UnHibernate}

{\bf Overview:} 
Arouse the specified VM from hibernation and revivify it

 \noindent {\bf Signature:} 
\begin{verbatim} void VM.UnHibernate (session_id s, vm_id vm)\end{verbatim}


\noindent{\bf Arguments:}

 
\vspace{0.3cm}
\begin{tabular}{|c|c|p{7cm}|}
 \hline
{\bf type} & {\bf name} & {\bf description} \\ \hline
{\tt vm\_id } & vm & The VM to unhibernate \\ \hline 

\end{tabular}

\vspace{0.3cm}

 \noindent {\bf Return Type:} 
{\tt 
void
}



\vspace{0.3cm}
\vspace{0.3cm}
\vspace{0.3cm}



\newpage\section{Host Management}

\subsection{Host.ListAllVMs}

{\bf Overview:} 
List all VMs installed on the specified host.

 \noindent {\bf Signature:} 
\begin{verbatim} List<vm_id> Host.ListAllVMs (session_id s)\end{verbatim}


\vspace{0.3cm}

 \noindent {\bf Return Type:} 
{\tt 
List<vm\_id>
}


List of VMs installed on specified host
\vspace{0.3cm}
\vspace{0.3cm}
\vspace{0.3cm}

\subsection{Host.ListRunningVMs}

{\bf Overview:} 
List all VMs currently running on the specified host.

 \noindent {\bf Signature:} 
\begin{verbatim} List<vm_id> Host.ListRunningVMs (session_id s)\end{verbatim}


\vspace{0.3cm}

 \noindent {\bf Return Type:} 
{\tt 
List<vm\_id>
}


List of VMs currently running on the specified host.
\vspace{0.3cm}
\vspace{0.3cm}
\vspace{0.3cm}

\subsection{Host.GetConfig}

{\bf Overview:} 
Get the host configuration as an XML string.

 \noindent {\bf Signature:} 
\begin{verbatim} XML Host.GetConfig (session_id s)\end{verbatim}


\vspace{0.3cm}

 \noindent {\bf Return Type:} 
{\tt 
XML
}


An XML string representing the host confiruation.
\vspace{0.3cm}
\vspace{0.3cm}
\vspace{0.3cm}

\subsection{Host.Disable}

{\bf Overview:} 
Puts the host into a state in which no new VMs can be started.

 \noindent {\bf Signature:} 
\begin{verbatim} void Host.Disable (session_id s)\end{verbatim}


\vspace{0.3cm}

 \noindent {\bf Return Type:} 
{\tt 
void
}



\vspace{0.3cm}
\vspace{0.3cm}
\vspace{0.3cm}

\subsection{Host.Shutdown}

{\bf Overview:} 
Shutdown the host. (This function can only be called if there are no currently running VMs ont he host and it is disabled.)

 \noindent {\bf Signature:} 
\begin{verbatim} void Host.Shutdown (session_id s)\end{verbatim}


\vspace{0.3cm}

 \noindent {\bf Return Type:} 
{\tt 
void
}



\vspace{0.3cm}
\vspace{0.3cm}
\vspace{0.3cm}

\subsection{Host.GetVersion}

{\bf Overview:} 
Return a string specifying version of Xen and support tools running on the host

 \noindent {\bf Signature:} 
\begin{verbatim} string Host.GetVersion (session_id s)\end{verbatim}


\vspace{0.3cm}

 \noindent {\bf Return Type:} 
{\tt 
string
}


Version string
\vspace{0.3cm}
\vspace{0.3cm}
\vspace{0.3cm}



\newpage\section{Task Management}

\subsection{Async.Task.GetStatus}

{\bf Overview:} 
Poll a running asynchronous RPC invocation and query its status

 \noindent {\bf Signature:} 
\begin{verbatim} XML Async.Task.GetStatus (session_id s, task_id task)\end{verbatim}


\noindent{\bf Arguments:}

 
\vspace{0.3cm}
\begin{tabular}{|c|c|p{7cm}|}
 \hline
{\bf type} & {\bf name} & {\bf description} \\ \hline
{\tt task\_id } & task & The ID of the RPC call to poll \\ \hline 

\end{tabular}

\vspace{0.3cm}

 \noindent {\bf Return Type:} 
{\tt 
XML
}


XML string describing status of specified asynchronous RPC invocation
\vspace{0.3cm}
\vspace{0.3cm}
\vspace{0.3cm}

\subsection{Async.Task.GetAllTasks}

{\bf Overview:} 
List all asynchronous RPC calls currently executing

 \noindent {\bf Signature:} 
\begin{verbatim} List<task_id> Async.Task.GetAllTasks (session_id s)\end{verbatim}


\vspace{0.3cm}

 \noindent {\bf Return Type:} 
{\tt 
List<task\_id>
}


A list of tasks currently executing. Note that
tasks are associated with users rather than sessions. Thus, if you logout and
login again with a different session but the same user, this function will still
return the user's running tasks.
\vspace{0.3cm}
\vspace{0.3cm}
\vspace{0.3cm}



